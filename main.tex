

\documentclass[10pt,a4paper]{altacv}

%% AltaCV uses the fontawesome and academicon fonts
%% and packages. 
%% See texdoc.net/pkg/fontawecome and http://texdoc.net/pkg/academicons for full list of symbols.
%% 
%% Compile with LuaLaTeX for best results. If you
%% want to use XeLaTeX, you'll need to install
%% Academicons.ttf in your operating system's font %% folder.


% Change the page layout if you need to
\geometry{left=1cm,right=9cm,marginparwidth=6.8cm,marginparsep=1.2cm,top=1cm,bottom=1cm}

% Change the font if you want to.
\setmainfont{Lato}

% Change the colours if you want to
\definecolor{VividPurple}{HTML}{3E0097}
\definecolor{SlateGrey}{HTML}{2E2E2E}
\definecolor{LightGrey}{HTML}{666666}
\definecolor{DarkBrown}{HTML}{3F2A14}
\definecolor{Gold}{HTML}{FFD700}
%\colorlet{heading}{VividPurple}
\colorlet{heading}{DarkBrown}
\colorlet{heading-line}{Gold}
\colorlet{accent}{DarkBrown}
\colorlet{emphasis}{SlateGrey}
\colorlet{body}{LightGrey}

% Change the bullets for itemize and rating marker
% for \cvskill if you want to
\renewcommand{\itemmarker}{{\small\textbullet}}
\renewcommand{\ratingmarker}{\faCircle}

%% sample.bib contains your publications
\addbibresource{sample.bib}

\begin{document}
\name{CASIMIR DÉSARMEAUX}
\tagline{Web Developement \& Data Science Freelancer}
% Cropped to square from my github page
\photo{2.5cm}{photo-pro.png}
\personalinfo{%
  % Not all of these are required!
  % You can add your own with \printinfo{symbol}{detail}
  \email{cdesarmeaux@loutech-software.com}
   \phone{+33 (0)7 85 23 49 27}
 % \mailaddress{Address, Street, 00000 County}
  \location{Paris, FR}
  \homepage{cdesarmeaux.com}

  \linkedin{linkedin.com/in/cdesarmeaux}
 \github{github.com/cdesarmeaux} 
%   \orcid{orcid.org/0000-0000-0000-0000} % Obviously making this up too
}

%% Make the header extend all the way to the right, if you want. Extend the right margin by 8cm (=6.8cm marginparwidth + 1.2cm marginparsep)
\begin{adjustwidth}{}{-8cm}
\makecvheader
\end{adjustwidth}

%% Provide the file name containing the sidebar contents as an optional parameter to \cvsection.
%% You can always just use \marginpar{...} if you do
%% not need to align the top of the contents to any
%% \cvsection title in the "main" bar.
\cvsection[page1sidebar]{Experience}

\cveventnew{Founder, Web Development \& Data Science Freelancer}{LouTech Software}{Built web apps, database structures, APIs for clients. Built in-house tools such as automatic project deployment and mail delivery systems.}{July 2017 -- Ongoing}{Paris, FR}

\divider

\cveventnew{Software Developer}{On Animations Studio}{Built web-based production management tools integrated with the in-house pipeline}{September 2016 -- June 2017}{Montreal, CA}

\divider

\cveventnew{Full Stack Web Developer, Data Scientist}{DBF Automobiles}{From the raw data of the salesmen activity (sales, offers, customer contacts), designed a web application to visualize this data, in real time.}{May 2015 -- July 2016}{Bordeaux, FR}

\divider

\cveventnew{Mobile Developer}{Herd Wisdom}{Designed and implemented the front-end of an an-app mobile game similar to Quiz up, within the Herd }{May 2014 -- July 2014}{Montreal, CA}

\divider

\cveventnew{Mobile Developer}{Wordsense}{Implemented the mobile front-end of an iphone game that uses financial market trading principals for betting on trending news events.}{May 2013 -- July 2013}{Paris, FR}


\cvsection{Education}

%\cvevent{M.S.\ in Computer Science}{Stanford University}{Sept 1997 -- June 1999}{}

%\divider

{\faGraduationCap} \cvevent{B.S.\ in Software Engineering}{McGill University, Faculty of Engineering}{September 2011 -- May 2016}{Montreal, CA}


\cvsection{PUBLICATIONS}

%\cvevent{M.S.\ in Computer Science}{Stanford University}{Sept 1997 -- June 1999}{}

%\divider

\cvachievement{The Dispersion of Build Maintenance Activity across Maven Lifecycle Phases}{International Conference on Mining Software Repositories, Mining challenge}{September 2011 -- May 2016}{Austin, US}
  
% \begin{itemize}
% \item Appointed by the founder Larry Page in 2001 to lead the Product Management and User Interaction teams
% \item Optimized Google's homepage and A/B tested every minor detail to increase usability (incl.~spacing between words, color schemes and pixel-by-pixel element alignment)
% \end{itemize}

% \divider

% \cvevent{Product Engineer}{Google}{23 June 1999 -- 2001}{Palo Alto, CA}

% \begin{itemize}
% \item Joined the company as employe \#20 and female employee \#1
% \item Developed targeted advertisement in order to use user's search queries and show them related ads}
% \end{itemize}

% \cvsection{A Day of My Life}

% % Adapted from @Jake's answer from http://tex.stackexchange.com/a/82729/226
% % \wheelchart{outer radius}{inner radius}{
% % comma-separated list of value/text width/color/detail}
% \wheelchart{1.5cm}{0.5cm}{ 10/13em/accent!30/Sleeping \& dreaming about work, 
%   25/9em/accent!60/Public resolving issues with Yahoo!\ investors,
%   5/12em/accent!10/New York \& San Francisco Ballet Jawbone board member, 
%   20/12em/accent!40/Spending time with family,
%   5/8em/accent!20/Business development for Yahoo!\ after the Verizon acquisition,
%   30/9em/accent/Showing Yahoo!\ employees that their work has meaning,
%   5/8em/accent!20/Baking cupcakes
% }

\clearpage



\end{document}
